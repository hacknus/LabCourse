\documentclass[a4paper,12pt]{article}
\usepackage[utf8]{inputenc}

\usepackage{amsmath}
\usepackage{amssymb}
\usepackage{graphicx}
\usepackage{subfig}
\usepackage{cancel} % Formeln kürzen
\usepackage{xcolor}
\usepackage{braket}
\usepackage{float}
\usepackage{siunitx}
\usepackage{placeins}
\usepackage[T1]{fontenc} % für matlab
\usepackage[framed, numbered]{matlab-prettifier} % für matlab

% read in csv tables
\usepackage{pgfplotstable,filecontents}
\pgfplotsset{compat=1.9}% supress warning
\usepackage{booktabs}
\pgfplotsset{
   /pgf/number format/textnumber/.style={
     fixed,
     fixed zerofill,
     precision=2,
     },
     /pgf/number format/textnumbertwo/.style={
     fixed,
     fixed zerofill,
     precision=4,
     },
}
\usepackage{datatool}



\usepackage[top=2.5cm, bottom=2.5cm, left=2.5cm, right=2.5cm]{geometry}
\usepackage{appendix}

\usepackage{biblatex}
\addbibresource{Literaturliste.bib}

\renewcommand*{\arraystretch}{1.3} % grössere Abstände in pmatrix

\usetikzlibrary{arrows,chains,matrix,positioning,scopes}


\tikzset{
 	>=triangle 45,
 	pos=.8,
 	photon/.style={decorate, thick,decoration={snake,segment length=5,post length=1mm,pre length = 1mm}},
	gluon/.style={decorate, thick, draw=black,
    		decoration={coil,amplitude=4pt, segment length=5pt,post length=1mm,pre length = 1mm}
	},
        particle/.style={ thick,
        postaction={decorate,
                    decoration={markings,mark=at position  0.5 with {\arrow[xshift=2pt +3\pgflinewidth]{<}}}
                   }
        },
        antiparticle/.style={ thick,
        postaction={decorate,
                    decoration={markings,mark=at position  0.5 with {\arrow[xshift=2pt +3\pgflinewidth]{>}}}
                   }
        }
}



% Default fixed font does not support bold face
\DeclareFixedFont{\ttb}{T1}{txtt}{bx}{n}{12} % for bold
\DeclareFixedFont{\ttm}{T1}{txtt}{m}{n}{12}  % for normal

% Custom colors
\usepackage{color}
\definecolor{deepblue}{rgb}{0,0,0.5}
\definecolor{deepred}{rgb}{0.6,0,0}
\definecolor{deepgreen}{rgb}{0,0.5,0}
\usepackage{hyperref}
\usepackage{listings}

% Python style for highlighting
\newcommand\pythonstyle{\lstset{
language=Python,
basicstyle=\ttm,
otherkeywords={self},             % Add keywords here
keywordstyle=\ttb\color{deepblue},
emph={MyClass,def,zeros,pass,sum,range,True,False},          % Custom highlighting
emphstyle=\ttb\color{deepred},    % Custom highlighting style
stringstyle=\color{deepgreen},
frame=tb,                         % Any extra options here
showstringspaces=false            % 
}}


% Python environment
\lstnewenvironment{python}[1][]
{
\pythonstyle
\lstset{#1}
}
{}



\begin{document}
    
    \pagenumbering{gobble} % keine Seitenzahl
    \begin{center}
    \vspace*{\fill} % zum vertikalen Zentrieren
    \LARGE{Lab Course II} \\
    \vspace{5mm}
    \Huge{Laser Gyroscope} \\
    \rule{10cm}{1pt} \\
    \vspace{2cm}
    \Large{Linus Stöckli \\Donat Hess} \\
    \vspace{1cm}
    \Large{Assistant: Florentin Spadin} \\
    \vspace{1cm}
    \Large{University of Bern, \\ October 2020}
    \vspace*{\fill} % zum vertikalen Zentrieren
\end{center}
    \newpage
    %\null \newpage % eine leere Seite
    
    
    
    \begin{abstract}
    \noindent 
    
    \end{abstract}
    \newpage
    
    \tableofcontents % Inhaltsverzeichnis
    \newpage
    
    \pagenumbering{arabic} % normale Seitenzahlen, beginnend mit Seite 1
    
    \section{Introduction} \label{sec:introduction}
    \FloatBarrier
    
    \section{Theory} \label{sec:theory}
        
    \FloatBarrier
    
    \newpage
\section{Method} \label{sec:method}
    \FloatBarrier
    
    \newpage
\section{Results and Discussion} \label{sec:results}
    \FloatBarrier

    \newpage
    \pagenumbering{gobble} % keine Seitenzahl
    
    \printbibliography %[title={Literaturverzeichnis}]
    
    \appendix
    \addappheadtotoc
    \newpage
    \pagenumbering{roman} % Seitenzahlen i, ii, iii, iv, etc.
    
   % \input{plots.tex}
    %\FloatBarrier
    \section{Results}
    \begin{table}[h!]
      \begin{center}
      \DTLsetseparator{,}
        \DTLloaddb[keys={omega,f,err}]{dat}{data.dat}
        \begin{tabular}{c|c}
            \toprule $\omega$ [\si{\deg\per\second}] & $f$ [\si{\hertz}]
            \DTLforeach{dat}{\o=omega,\f=f,\err=err}
            {\DTLiffirstrow{\\ \midrule}{\\}
            \o & \pgfmathprintnumber[textnumber]\f~$\pm$~\pgfmathprintnumber[textnumber]\err }
            \\\bottomrule
        \end{tabular}
        \caption{Data}
      \end{center}
    \end{table}
    \section{Python Scripts}
    
    \begin{python}

    \end{python}

  

    \FloatBarrier
\end{document}